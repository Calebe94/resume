\documentclass[11pt,a4paper,sans]{moderncv}

\moderncvstyle{casual}
\moderncvcolor{blue}

\usepackage[utf8]{inputenc}
\usepackage[scale=0.85]{geometry}
\nopagenumbers

% dados pessoais
\name{Edimar Calebe}{Castanho}
\title{Desenvolvedor de Software}
\address{Brasil}{Curitiba, Paraná}
\email{contato@calebe.dev.br}
\phone[mobile]{+55~(41)998271302}
\social[linkedin]{calebe94}
\social[github]{calebe94}
\extrainfo{\faRss\href{https://blog.calebe.dev.br/}{blog.calebe.dev.br}}

\begin{document}

\maketitle

\section{Resumo}
\cvitem{}{
Graduado em Engenharia de Computação com experiência em sistemas embarcados e programação de baixo nível. Conhecimento sólido em linguagens como C e Python, além de experiência em sistemas operacionais Linux e FreeRTOS. Proficiente em ferramentas de DevOps e controle de versão como Git, Jira, Github e Gitlab. Profissional dedicado, capaz de enfrentar desafios e trabalhar em equipe.}

\section{Experiência Profissional}
\cventry{2020 -- Presente}{Analista de Sistemas Embarcados}{Grupo T2I}{Curitiba - PR}{}{
\begin{itemize}
\item Desenvolvimento e manutenção de sistemas embarcados usando C, Python e FreeRTOS.
\item Colaboração com equipes de hardware e software para integração do sistema.
\item Implementação de testes unitários e integração contínua (CI/CD) para garantir a qualidade do software.
\item Utilização do Git para controle de versão e gestão de tarefas.
\end{itemize}
}

\cventry{2019 -- 2020}{Estagiário de Sistemas Embarcados}{Alta Rail Technology - ART}{Colombo - PR}{}{
\begin{itemize}
\item Participação no desenvolvimento de sistemas embarcados com foco em C e Linux.
\item Colaboração com a equipe de engenharia para testes e depuração de hardware.
\item Realização de testes de software e suporte na resolução de problemas.
\item Utilização do Git e Jira para controle de versão e gestão de tarefas.
\end{itemize}
}

\cventry{2017 -- 2019}{Estagiário de Sistemas Embarcados}{Grupo T2I}{Curitiba - PR}{}{
\begin{itemize}
\item Participação no desenvolvimento de sistemas embarcados com foco em C e FreeRTOS.
\end{itemize}
}

\section{Habilidades Técnicas}
\cvitem{Linguagens de Programação}{C, Python, Shell Script, Javascript, HTML, Flutter, Flask, CSS}
\cvitem{Sistemas Operacionais}{Debian, Linux, FreeRTOS}
\cvitem{Ferramentas}{Git, Jira, Github, Gitlab, Docker, SDL2}
\cvitem{Metodologias}{Testes Unitários, DevOps, Pipelines, CI/CD}
\cvitem{Idiomas}{Português (Nativo), Inglês (Fluente)}

\section{Educação}
\cventry{2015 -- Presente}{Engenharia de Computação}{Universidade Positivo}{Curitiba}{}{}
\cventry{2011 -- 2013}{Curso Técnico Eletromecânico}{Instituto Federal do Paraná - IFPR}{Curitiba}{}{}

\section{Projetos Pessoais}
\cvitem{\href{https://blog.calebe.dev.br/posts/teclado-appa.html}{Teclado Appa}}{Um teclado ortolinear 4x12 baseado em Atmega328p com VUSB e firmware QMK, feito apenas com componentes through-hole (furos passantes).}

\cvitem{\href{https://TinyToolSH.github.io}{TinyTools}}{Ferramentas pequenas escritas em POSIX Shell Script para melhorar a produtividade de forma simples.}

\end{document}
