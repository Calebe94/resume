\documentclass[11pt,a4paper,sans]{moderncv}

\moderncvstyle{casual}
\moderncvcolor{blue}

\usepackage[utf8]{inputenc}
\usepackage[scale=0.85]{geometry}
\nopagenumbers

% dados pessoais
\name{Edimar Calebe}{Castanho}
\title{Desenvolvedor Sênior em Python e Cloud}
\address{Brasil}{Curitiba, Paran\'a}
\email{contato@calebe.dev.br}
\phone[mobile]{+55~(41)~99827~1302}
\social[linkedin]{calebe94}
\social[github]{calebe94}
\extrainfo{\faRss\href{https://blog.calebe.dev.br/}{blog.calebe.dev.br}}

\begin{document}

\maketitle

\section{Resumo}
\cvitem{}{
Sou desenvolvedor sênior com forte foco em Python e computação em nuvem. Tenho experiência em AWS, GCP e Magalu Cloud, e sou especialista em automação em nuvem, microsserviços escaláveis e engenharia de dados. Tenho habilidade comprovada em projetar e otimizar soluções de alto desempenho com frameworks, ferramentas e arquiteturas modernas.
}

\section{Experiência Profissional}
\cventry{2023 -- Presente}{Desenvolvedor Sênior em Cloud}{LuizaLabs - Magalu Cloud}{Curitiba - PR}{}{
\begin{itemize}
\item Desenvolvi ferramentas de automação em nuvem na tribo IAAS, otimizando a gestão de infraestrutura na Magalu Cloud.
\item Criei frameworks em Python para microsserviços escaláveis usando FastAPI e Django.
\item Melhorei o desempenho e a confiabilidade da infraestrutura em nuvem utilizando Openstack, Juju e Ansible.
\item Integrei serviços da AWS com ferramentas internas para otimizar pipelines de processamento de dados.
\item Tecnologias: Python, FastAPI, Snowflake, RabbitMQ, Openstack, AWS, Docker, GitLab CI/CD.
\end{itemize}
}

\cventry{2020 -- 2023}{Analista de Cloud e Sistemas Embarcados}{T2I Group}{Curitiba - PR}{}{
\begin{itemize}
\item Migrei sistemas embarcados legados para soluções modernas baseadas em Python integradas à nuvem.
\item Automatizei fluxos de trabalho ETL e otimizei pipelines de dados usando AWS Glue e Lambda.
\item Implementei aplicações conteinerizadas para implantação em nuvem com Docker e Kubernetes.
\item Desenvolvi arquiteturas serverless e integrei Python com diversos serviços em nuvem para automação de ponta a ponta.
\item Tecnologias: Python, AWS Glue, Lambda, Docker, Kubernetes, FreeRTOS, C++.
\end{itemize}
}

\cventry{2019 -- 2020}{Estagiário em Sistemas Embarcados}{Alta Rail Technology - ART}{Colombo - PR}{}{
\begin{itemize}
\item Automatizei testes para sistemas embarcados com Python, aumentando a eficiência da validação.
\item Integrei soluções em nuvem aos fluxos de trabalho existentes, utilizando AWS para escalabilidade.
\item Desenvolvi utilitários para interação com bancos de dados e integração hardware-software.
\item Tecnologias: Python, SQLite, AWS, C++, Git.
\end{itemize}
}

\section{Habilidades Técnicas}
\cvitem{Programação}{Python, SQL, C++, Shell Script, Flask, Django, FastAPI}
\cvitem{Plataformas de Nuvem}{AWS (Glue, Lambda, CloudFormation, CDK), GCP, Openstack, Magalu Cloud, Docker}
\cvitem{Bancos de Dados}{Snowflake, PostgreSQL, MySQL, Elasticsearch, SQLite}
\cvitem{DevOps}{CI/CD (GitLab, Jenkins), Docker, Ansible, Juju, Terraform}
\cvitem{Especialidades}{Automação em Nuvem, Engenharia de Dados, Microsserviços Escaláveis, Sistemas Orientados a Eventos, Arquiteturas Serverless}
\cvitem{Idiomas}{Português (Nativo), Inglês (Fluente)}

\section{Formação}
\cventry{2015 -- 2023}{Engenharia da Computação}{Universidade Positivo}{Curitiba}{}{}
\cventry{2011 -- 2013}{Curso Técnico em Eletromecânica}{Instituto Federal do Paran\'a - IFPR}{Curitiba}{}{}

\end{document}
