\documentclass[11pt,a4paper,sans]{moderncv}

\moderncvstyle{classic}
\moderncvcolor{blue}

\usepackage[utf8]{inputenc}
\usepackage[scale=0.85]{geometry}
\nopagenumbers

% personal data
\name{Edimar Calebe}{Castanho}
\title{Senior Software Engineer \& Cloud Architect}
\address{Curitiba, Brazil}{Open to relocation (Europe \& US)}
\email{contato@calebe.dev.br}
\phone[mobile]{+55~(41)~99827~1302}
\social[linkedin]{calebe94}
\social[github]{calebe94}
\extrainfo{\faRss\href{https://blog.calebe.dev.br/}{blog.calebe.dev.br}}

\begin{document}

\maketitle

\section{Summary}
\cvitem{}{
Senior Software Engineer and Cloud Architect combining deep embedded systems experience with modern cloud-native engineering and DevOps practices. Track record designing and shipping end-to-end solutions — from realtime firmware (ESP32, FreeRTOS, TM4C, PIC) and IoT gateways to backend services and infrastructure automation (Python, FastAPI/Flask, MQTT, SNMP). Strong experience with Infrastructure as Code, CI/CD and GitOps (Terraform, Ansible, GitLab/GitHub Actions, ArgoCD), virtualization and IaaS (OpenStack, KVM, Incus/LXD), and container platforms (Docker, Kubernetes). Experienced in developer tooling, image engineering and automation of deployment pipelines to improve reliability and time-to-production. Fluent in English and Portuguese; targeting senior engineering roles in Europe and the US.
}

\section{Professional Experience}
\cventry{2023 -- Present}{Senior Cloud Engineer}{LuizaLabs (Magalu Cloud)}{Curitiba, Brazil}{}{
\begin{itemize}
\item Led the design and automation of virtual machine provisioning workflows using OpenStack (Yoga) and Incus, ensuring high reliability and scalability in Magalu Cloud's IaaS platform.
\item Developed Python-based automation tools leveraging asyncio, Pydantic, and pytest to manage VM lifecycle operations (start, stop, restart, freeze, unfreeze) and integrate with internal APIs.
\item Created and maintained custom Ubuntu and Windows Server images, implementing automated SQL Server installation with Ansible and Cloudbase-init, as well as image generalization workflows to streamline deployments.
\item Designed Terraform modules for provisioning server-client benchmarking environments, integrating sysbench, fio, and iperf to test CPU, memory, disk, and network performance.
\item Researched and implemented optimizations for KVM-based workloads, focusing on performance tuning, image compression, and storage efficiency with QCOW2.
\item Enhanced OpenStack Glance and Nova workflows to reduce image deployment times and operational overhead.
\item Integrated image build pipelines with CI/CD workflows, ensuring consistent, reproducible, and compliant image releases across multiple regions.
\item Collaborated with security teams to proactively address OS vulnerabilities across Linux distributions and Windows Server versions.
\item Conducted API comparisons between Magalu Cloud and Incus to guide platform interoperability and feature parity strategies.
\item Provided technical guidance and mentorship on cloud image engineering, OpenStack, and virtualization best practices.
\item \textbf{Technologies:} Python, asyncio, Pydantic, pytest, OpenStack (Yoga: Glance, Nova, Cinder, Neutron), Incus, KVM/QEMU, Ansible, Cloudbase-init, Terraform, Juju, Docker, GitLab CI/CD, QCOW2, sysbench, fio, iperf, AWS (EC2, S3, Lambda)
\end{itemize}
}

\cventry{2020 -- 2023}{Embedded Systems Analyst}{T2I Group}{Curitiba, Brazil}{}{
\begin{itemize}
\item Led the development of a \textbf{telemetry system for highway weather stations}, implementing the SNMP protocol (OID tree management) on ESP32 microcontrollers via a custom fork of the GNU LwIP stack.
\item Designed and built cross-platform mobile applications in \textbf{Flutter} for Android and iOS, delivering Material Design-compliant UIs and integrating real-time IoT data feeds.
\item Architected and developed \textbf{MIP V2}, a vehicle detection and classification system for highways, combining Texas Instruments TM4C microcontrollers with ESP32 modules to collect sensor data, identify vehicle types, and detect axle configurations.
\item Built the embedded firmware in \textbf{C/C++ (FreeRTOS, ESP-IDF)} and designed server-side components using a custom Debian-based Linux distribution (“T2Ibian”) with pre-seeded installation for rapid deployment.
\item Implemented the \textbf{data aggregation server} in Python with Flask, using Mosquitto (MQTT) for message brokering between roadside units and central systems.
\item Migrated and consolidated all company repositories to \textbf{GitLab} with self-hosted runners, establishing CI/CD pipelines for automated builds and deployments; later transitioned infrastructure to GitHub Actions.
\item Utilized a diverse stack including \textbf{Python, Docker, AWS, GitLab CI/CD, GitHub Actions, Jenkins, C/C++, Flask} to deliver production-ready embedded and cloud-integrated solutions.
\end{itemize}
}

\cventry{2019 -- 2020}{Embedded Systems Intern}{Alta Rail Technology (ART)}{Colombo, Brazil}{}{
\begin{itemize}
\item Developed a \textbf{Python + Flask dashboard} to display real-time train operational data, including wagon count, per-wagon sensor status, and temperature readings, directly to locomotive operators. Integrated with proprietary embedded hardware (SOBC and HMI) as part of the company’s commercial offering.
\item Migrated to \textbf{C\#} development for Windows Server-based router software, implementing an optimized communication library that reduced satellite transmission costs by minimizing redundant OSI layer headers.
\item Collaborated on embedded systems software for train control hardware, working with \textbf{FreeRTOS, Linux, and Yocto Project} to enhance reliability and maintainability.
\item Conducted unit and integration testing using \textbf{VectorCast} and Python test scripts, ensuring compliance with railway safety and performance standards.
\item Contributed to cross-functional teams to deliver hardware-software integration projects, balancing performance, resource constraints, and communication protocol efficiency.
\end{itemize}
}

\cventry{2017}{Embedded Systems Intern}{T2I Group}{Curitiba, Brazil}{}{
\begin{itemize}
\item Designed and implemented an \textbf{OBD-II automotive fleet monitoring device} using PIC18F microcontrollers, proprietary vehicle interface chips, GSM/GPRS, and GPS/GNSS modules to collect and transmit real-time vehicle diagnostics and geolocation data to a REST API.
\item Developed IoT solutions leveraging \textbf{Arduino, Bluetooth Low Energy (BLE) beacons}, and GPRS integration for museum exhibit tracking and interactive visitor experiences.
\item Built an \textbf{aquarium automation system} on a Raspberry Pi 3-based SBC running Linux, integrating Arduino controllers to schedule lighting, regulate intensity, and automate environmental events via a web-based dashboard (Node.js + Vue.js).
\item Programmed embedded firmware in \textbf{C/C++ (AVR, ATmega328, PIC18F)} and Python, integrating FreeRTOS for real-time control and peripheral management.
\item Utilized modern development tools and workflows including Docker, Vagrant, Geany, GTK+, and Qt for cross-platform application development.
\item Collaborated in a multidisciplinary engineering team, following structured development processes to deliver reliable embedded and IoT solutions for commercial applications.
\end{itemize}
}

\section{Technical Skills}
\cvitem{Core Languages}{Python, C/C++, SQL, Bash / Shell}
\cvitem{Web \& APIs}{FastAPI, Flask, Django, RESTful APIs, Node.js, Vue.js}
\cvitem{Cloud \& Virtualization}{OpenStack (Glance, Nova, Cinder, Neutron), AWS (EC2, S3, Lambda, Glue, CDK), GCP, Incus (LXD), KVM/QEMU}
\cvitem{Containers \& Orchestration}{Docker, Kubernetes, Docker Compose}
\cvitem{IaC \& Configuration}{Terraform, Ansible, Juju}
\cvitem{CI/CD \& GitOps}{GitLab CI/CD / GitLab Actions, GitHub Actions, Jenkins, ArgoCD, GitLab runners}
\cvitem{Embedded \& IoT}{ESP32 (ESP-IDF), FreeRTOS, TM4C (Tiva C), PIC18F, ATmega328 (AVR), Arduino, Yocto Project, Linux (Debian/T2Ibian)}
\cvitem{Networking \& Protocols}{MQTT (Mosquitto), SNMP, OBD-II, GSM/GPRS, GPS/GNSS, BLE, OSI-layer optimizations}
\cvitem{Data \& Databases}{PostgreSQL, MySQL, SQLite, Snowflake, Elasticsearch}
\cvitem{Testing \& Observability}{pytest, VectorCast, sysbench, fio, iperf}
\cvitem{Tools \& Utilities}{Git, Docker, Vagrant, Cloudbase-init, cloud-init, Geany, Qt, GTK+}
\cvitem{Methodologies \& Expertise}{Infrastructure as Code, Cloud Automation, Scalable Microservices, DevOps, GitOps, Agile / Scrum}
\cvitem{Languages}{Portuguese (Native), English (Fluent)}

\section{Education}
\cventry{2015 -- 2023}{Computer Engineering}{Universidade Positivo}{Curitiba, Brazil}{}{}
\cventry{2011 -- 2013}{Electromechanical Technical Course}{Instituto Federal do Paran\'a (IFPR)}{Curitiba, Brazil}{}{}

\end{document}
