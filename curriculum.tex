
\documentclass[11pt,a4paper,sans]{moderncv}

\moderncvstyle{classic}
\moderncvcolor{blue}

\usepackage[utf8]{inputenc}
\usepackage[scale=0.85]{geometry}
\nopagenumbers

% personal data
\name{Edimar Calebe}{Castanho}
\title{Senior Software Engineer \& Cloud Architect}
\address{Curitiba, Brazil}{Open to relocation (Europe)}
\email{contato@calebe.dev.br}
\phone[mobile]{+55~(41)~99827~1302}
\social[linkedin]{calebe94}
\social[github]{calebe94}
\extrainfo{\faRss\href{https://blog.calebe.dev.br/}{blog.calebe.dev.br}}

\begin{document}

\maketitle

\section{Summary}
\cvitem{}{
Senior Software Engineer with extensive experience in Python and cloud computing, specializing in scalable microservices and data pipelines.
Skilled in AWS, GCP, OpenStack, Incus (LXD), with expertise in Infrastructure as Code (Terraform, Ansible) and DevOps practices (CI/CD with GitLab and Jenkins).
Proven ability to lead the development of high-performance, cloud-native solutions using modern frameworks (FastAPI, Django) and containerization.
Adept at collaborating with cross-functional teams to align technical strategies with business objectives.
Fluent in English and Portuguese, open to opportunities in Europe and the US.
}

\section{Professional Experience}
\cventry{2023 -- Present}{Senior Cloud Developer}{LuizaLabs (Magalu Cloud)}{Curitiba, Brazil}{}{
\begin{itemize}
\item Spearheaded the development of cloud automation tools within the IaaS team, streamlining infrastructure management on Magalu Cloud.
\item Designed and implemented Python-based frameworks for scalable microservices using FastAPI and Django.
\item Enhanced cloud infrastructure performance and reliability using OpenStack, Juju, Terraform, and Ansible.
\item Worked extensively with AWS services (Lambda, Glue, CDK) to optimize data processing pipelines and enable serverless integrations.
\item Implemented GitOps workflows with ArgoCD to automate Kubernetes application deployments.
\item Mentored junior engineers and collaborated with cross-functional teams to align technical solutions with business objectives.
\item \textbf{Technologies:} Python, FastAPI, Django, Snowflake, RabbitMQ, OpenStack, AWS (Lambda, Glue, CDK), Docker, Kubernetes, GitLab CI/CD (Actions), Terraform, Ansible, ArgoCD
\end{itemize}
}

\cventry{2020 -- 2023}{Embedded Systems Analyst}{T2I Group}{Curitiba, Brazil}{}{
\begin{itemize}
\item Led the development of a \textbf{telemetry system for highway weather stations}, implementing the SNMP protocol (OID tree management) on ESP32 microcontrollers via a custom fork of the GNU LwIP stack.
\item Designed and built cross-platform mobile applications in \textbf{Flutter} for Android and iOS, delivering Material Design-compliant UIs and integrating real-time IoT data feeds.
\item Architected and developed \textbf{Mipe V2}, a vehicle detection and classification system for highways, combining Texas Instruments TM4C microcontrollers with ESP32 modules to collect sensor data, identify vehicle types, and detect axle configurations.
\item Built the embedded firmware in \textbf{C/C++ (FreeRTOS, ESP-IDF)} and designed server-side components using a custom Debian-based Linux distribution (“T2Ibian”) with pre-seeded installation for rapid deployment.
\item Implemented the \textbf{data aggregation server} in Python with Flask, using Mosquitto (MQTT) for message brokering between roadside units and central systems.
\item Migrated and consolidated all company repositories to \textbf{GitLab} with self-hosted runners, establishing CI/CD pipelines for automated builds and deployments; later transitioned infrastructure to GitHub Actions.
\item Utilized a diverse stack including \textbf{Python, Docker, AWS, GitLab CI/CD, GitHub Actions, Jenkins, C/C++, Flask} to deliver production-ready embedded and cloud-integrated solutions.
\end{itemize}
}

\cventry{2019 -- 2020}{Embedded Systems Intern}{Alta Rail Technology (ART)}{Colombo, Brazil}{}{
\begin{itemize}
\item Developed a \textbf{Python + Flask dashboard} to display real-time train operational data, including wagon count, per-wagon sensor status, and temperature readings, directly to locomotive operators. Integrated with proprietary embedded hardware (SOBC and HMI) as part of the company’s commercial offering.
\item Migrated to \textbf{C\#} development for Windows Server-based router software, implementing an optimized communication library that reduced satellite transmission costs by minimizing redundant OSI layer headers.
\item Collaborated on embedded systems software for train control hardware, working with \textbf{FreeRTOS, Linux, and Yocto Project} to enhance reliability and maintainability.
\item Conducted unit and integration testing using \textbf{VectorCast} and Python test scripts, ensuring compliance with railway safety and performance standards.
\item Contributed to cross-functional teams to deliver hardware-software integration projects, balancing performance, resource constraints, and communication protocol efficiency.
\end{itemize}
}

\cventry{2017}{Embedded Systems Intern}{T2I Group}{Curitiba, Brazil}{}{
\begin{itemize}
\item Designed and implemented an \textbf{OBD-II automotive fleet monitoring device} using PIC18F microcontrollers, proprietary vehicle interface chips, GSM/GPRS, and GPS/GNSS modules to collect and transmit real-time vehicle diagnostics and geolocation data to a REST API.
\item Developed IoT solutions leveraging \textbf{Arduino, Bluetooth Low Energy (BLE) beacons}, and GPRS integration for museum exhibit tracking and interactive visitor experiences.
\item Built an \textbf{aquarium automation system} on a Raspberry Pi 3-based SBC running Linux, integrating Arduino controllers to schedule lighting, regulate intensity, and automate environmental events via a web-based dashboard (Node.js + Vue.js).
\item Programmed embedded firmware in \textbf{C/C++ (AVR, ATmega328, PIC18F)} and Python, integrating FreeRTOS for real-time control and peripheral management.
\item Utilized modern development tools and workflows including Docker, Vagrant, Geany, GTK+, and Qt for cross-platform application development.
\item Collaborated in a multidisciplinary engineering team, following structured development processes to deliver reliable embedded and IoT solutions for commercial applications.
\end{itemize}
}

\section{Technical Skills}
\cvitem{Programming Languages}{Python, C++, SQL, Bash/Shell}
\cvitem{Frameworks \& Libraries}{FastAPI, Django, Flask}
\cvitem{Cloud Platforms}{AWS (Lambda, Glue, CDK), GCP, OpenStack}
\cvitem{Databases \& Data}{Snowflake, PostgreSQL, MySQL, Elasticsearch, SQLite}
\cvitem{DevOps \& Tools}{GitLab Actions, Jenkins, Docker, Kubernetes, Terraform, Ansible, ArgoCD}
\cvitem{Other Tools}{Git, CI/CD Pipelines, Docker Compose}
\cvitem{Expertise}{Infrastructure as Code, Cloud Automation, Scalable Microservices, Data Engineering, Agile \& Scrum}
\cvitem{Languages}{Portuguese (Native), English (Fluent)}

\section{Education}
\cventry{2015 -- 2023}{Computer Engineering}{Universidade Positivo}{Curitiba, Brazil}{}{}
\cventry{2011 -- 2013}{Electromechanical Technical Course}{Instituto Federal do Paran\'a (IFPR)}{Curitiba, Brazil}{}{}

\end{document}
